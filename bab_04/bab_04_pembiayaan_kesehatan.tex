\chapter{PEMBIAYAAN KESEHATAN}
Lorem ipsum dolor sit amet, consectetur adipiscing elit, sed do eiusmod tempor incididunt ut labore et dolore magna aliqua. Ut enim ad minim veniam, quis nostrud exercitation ullamco laboris nisi ut aliquip ex ea commodo consequat. Duis aute irure dolor in reprehenderit in voluptate velit esse cillum dolore eu fugiat nulla pariatur. Excepteur sint occaecat cupidatat non proident, sunt in culpa qui officia deserunt mollit anim id est laborum.

\section[PEMBIAYAAN MASYARAKAT]{PEMBIAYAAN KESEHATAN OLEH MASYARAKAT}
Sed ut perspiciatis unde omnis iste natus error sit voluptatem accusantium doloremque laudantium, totam rem aperiam, eaque ipsa quae ab illo inventore veritatis et quasi architecto beatae vitae dicta sunt explicabo. Nemo enim ipsam voluptatem quia voluptas sit aspernatur aut odit aut fugit, sed quia consequuntur magni dolores eos qui ratione voluptatem sequi nesciunt.

Cakupan jaminan kesehatan melalui BPJS Kesehatan di \namaKabupaten pada tahun \tP adalah sebesar XX,XX\%, di mana YY,YY\% dari jumlah tersebut adalah Penerima Bantuan Iuran (PBI) bersumber APBD dan APBN (\autoref{fig:Cakupan-BPJS}). Diperkirakan terdapat XX.YYY penduduk dari total YY.ZZZ penduduk yang masih belum mendapat perlindungan jaminan kesehatan atau lebih memilih menggunakan asuransi kesehatan swasta (\autoref{fig:Cakupan-Jamkes}).

\begin{figure}[H]
    \centering{}
%    \includegraphics[width=0.6\textwidth]{bab_04/bab_04_01_jaminanKesehatan_a}
    \caption{Cakupan BPJS Kesehatan \namaKabupaten Tahun \tP}
    \label{fig:Cakupan-BPJS}
\end{figure}

\begin{figure}[H]
    \centering{}
%    \includegraphics[width=0.4\textwidth]{bab_04/bab_04_01_jaminanKesehatan_b}
    \caption{Cakupan BPJS Kesehatan \namaKabupaten Tahun \tP}
    \label{fig:Cakupan-Jamkes}
\end{figure}


\section[PEMBIAYAAN PEMERINTAH]{PEMBIAYAAN KESEHATAN OLEH PEMERINTAH}
\subsection{Pembiayaan melalui Anggaran Pendapatan dan Belanja Daerah}
Alokasi Anggaran Kesehatan di \namaKabupaten pada tahun
\tP melalui APBD \namaKabupaten Tahun \tP (mencakup anggaran
Dinas Kesehatan dan UPTD Kesehatan) adalah sebesar
Rp  153.693.125.075,- atau 20,09\% dari jumlah belanja APBD \namaKabupaten Tahun \tP. Selain itu terdapat anggaran belanja bersumber
APBN sebesar Rp 33.998.138.000,- berupa Dana Alokasi Khusus (DAK)
fisik dan  Rp 7.375145.000,- berupa dana DAK nonfisik. Dengan demikian total anggaran kesehatan \namaKabupaten
Tahun \tP adalah Rp. 195.066.408.075,- atau Rp. 1.593.705,85 per
kapita.

\begin{figure}[htb]
	\centering{}
%	\includegraphics[width=0.9\textwidth]{bab_04/bab_04_03_anggaranKesehatan}
	\caption{Persentase Anggaran Kesehatan Kab. Belitung Timur Tahun \tP}
	\label{fig:Anggaran-Kesehatan}
\end{figure}

\subsection{Pembiayaan Jaminan Kesehatan Masyarakat Pada Anggaran Dinas Kesehatan}
Lorem ipsum dolor sit amet, consectetur adipiscing elit, sed do eiusmod tempor incididunt ut labore et dolore magna aliqua. Ut enim ad minim veniam, quis nostrud exercitation ullamco laboris nisi ut aliquip ex ea commodo consequat. Duis aute irure dolor in reprehenderit in voluptate velit esse cillum dolore eu fugiat nulla pariatur. Excepteur sint occaecat cupidatat non proident, sunt in culpa qui officia deserunt mollit anim id est laborum.

Sed ut perspiciatis unde omnis iste natus error sit voluptatem accusantium doloremque laudantium, totam rem aperiam, eaque ipsa quae ab illo inventore veritatis et quasi architecto beatae vitae dicta sunt explicabo. Nemo enim ipsam voluptatem quia voluptas sit aspernatur aut odit aut fugit, sed quia consequuntur magni dolores eos qui ratione voluptatem sequi nesciunt.

\begin{figure}[htb]
	\centering{}
%	\includegraphics[width=0.9\textwidth]{bab_04/bab_04_04_proporsiPBI}
	\caption{Proporsi PBI terhadap Anggaran DKPPKB Kab. Belitung Timur Tahun \tP}
	\label{fig:Proporsi-PBI}
\end{figure}

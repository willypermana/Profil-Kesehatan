\chapter{SARANA PRASARANA KESEHATAN}
Paragraf sarana prasarana kesehatan. Lorem ipsum dolor sit amet, consectetur adipiscing elit, sed do eiusmod tempor incididunt ut labore et dolore magna aliqua. Ut enim ad minim veniam, quis nostrud exercitation ullamco laboris nisi ut aliquip ex ea commodo consequat. Duis aute irure dolor in reprehenderit in voluptate velit esse cillum dolore eu fugiat nulla pariatur. Excepteur sint occaecat cupidatat non proident, sunt in culpa qui officia deserunt mollit anim id est laborum.

\section{FASILITAS PELAYANAN KESEHATAN}
Lorem ipsum dolor sit amet, consectetur adipiscing elit, sed do eiusmod tempor incididunt ut labore et dolore magna aliqua. Ut enim ad minim veniam, quis nostrud exercitation ullamco laboris nisi ut aliquip ex ea commodo consequat. Duis aute irure dolor in reprehenderit in voluptate velit esse cillum dolore eu fugiat nulla pariatur. Excepteur sint occaecat cupidatat non proident, sunt in culpa qui officia deserunt mollit anim id est laborum.

\subsection{Rumah Sakit}
Curabitur pretium tincidunt lacus. Nulla gravida orci a odio. Nullam varius, turpis et commodo pharetra, est eros bibendum elit, nec luctus magna felis sollicitudin mauris. Integer in mauris eu nibh euismod gravida. Duis ac tellus et risus vulputate vehicula. Donec lobortis risus a elit. Etiam tempor. Ut ullamcorper, ligula eu tempor congue, eros est euismod turpis, id tincidunt sapien risus a quam. Maecenas fermentum consequat mi. 

Donec fermentum. Pellentesque malesuada nulla a mi. Duis sapien sem, aliquet nec, commodo eget, consequat quis, neque. Aliquam faucibus, elit ut dictum aliquet, felis nisl adipiscing sapien, sed malesuada diam lacus eget erat. Cras mollis scelerisque nunc. Nullam arcu. Aliquam consequat. Curabitur augue lorem, dapibus quis, laoreet et, pretium ac, nisi. Aenean magna nisl, mollis quis, molestie eu, feugiat in, orci. In hac habitasse platea dictumst:
\begin{enumerate}
  \item Lorem ipsum dolor sit amet, consectetuer adipiscing elit.;
  \item Aliquam tincidunt mauris eu risus; and
  \item Vestibulum auctor dapibus neque.
\end{enumerate}

Jumlah Rumah Sakit di \nama_kabupaten pada tahun \tP adalah
sebanyak x (...) unit, yaitu RS

\subsection{Puskesmas}
Curabitur pretium tincidunt lacus. Nulla gravida orci a odio. Nullam varius, turpis et commodo pharetra, est eros bibendum elit, nec luctus magna felis sollicitudin mauris. Integer in mauris eu nibh euismod gravida. Duis ac tellus et risus vulputate vehicula. Donec lobortis risus a elit. Etiam tempor. Ut ullamcorper, ligula eu tempor congue, eros est euismod turpis, id tincidunt sapien risus a quam. Maecenas fermentum consequat mi.

Jumlah Puskesmas di \nama_kabupaten pada tahun
\tP adalah sebanyak x (...) unit Puskesmas dengan rincian x (...)
unit Puskesmas Keperawatan yaitu:
\begin{enumerate}
  \item Puskesmas A;
  \item Puskesmas B;
  \item Puskesmas dst...;and
  \item Puskesmas C.
\end{enumerate}
Sedangkan x (...) unit Non Keperawatan adalah:
\begin{enumerate}
  \item Puskesmas A;
  \item Puskesmas B;
  \item Puskesmas dst...;and
  \item Puskesmas C.
\end{enumerate}

\subsection{Puskesmas Pembantu}

Donec fermentum. Pellentesque malesuada nulla a mi. Duis sapien sem, aliquet nec, commodo eget, consequat quis, neque. Aliquam faucibus, elit ut dictum aliquet, felis nisl adipiscing sapien, sed malesuada diam lacus eget erat. Cras mollis scelerisque nunc. Nullam arcu. Aliquam consequat. Curabitur augue lorem, dapibus quis, laoreet et, pretium ac, nisi. Aenean magna nisl, mollis quis, molestie eu, feugiat in, orci. In hac habitasse platea dictumst.

Jumlah Puskesmas Pembantu di \nama_kabupaten pada tahun \tP
adalah sebanyak x (...) Pustu (\autoref{tab:Puskemas-dan-Pustu}).

\begin{table}[!ht]
\caption{Puskemas dan Jumlah Puskesmas Pembantu di \nama_kabupaten Tahun \tP}
\label{tab:Puskemas-dan-Pustu}
\centering{}%
\ra{1.3}

\begin{tabular}{cllc}
\toprule
No & Kecamatan & \multicolumn{1}{l}{Puskesmas} & Jumlah Puskesmas Pembantu\\
\midrule
1. & Kecamatan A & Puskesmas A & 1\\
\rowcolor{black!20}2. & Kecamatan B & Puskesmas B & 2\\
3. & Kecamatan ... & Puskesmas ... & 3\\
\midrule
\rowcolor{blue!20}\multicolumn{2}{c}{Jumlah} & \multicolumn{1}{c}{7} & X\\
\bottomrule
\end{tabular}
\end{table}

\section{AKSES DAN MUTU PELAYANAN KESEHATAN}
\subsection{Kunjungan rawat jalan dan rawat inap}
Lorem ipsum dolor sit amet, consectetur adipiscing elit, sed do eiusmod tempor incididunt ut labore et dolore magna aliqua. Ut enim ad minim veniam, quis nostrud exercitation ullamco laboris nisi ut aliquip ex ea commodo consequat. Duis aute irure dolor in reprehenderit in voluptate velit esse cillum dolore eu fugiat nulla pariatur. Excepteur sint occaecat cupidatat non proident, sunt in culpa qui officia deserunt mollit anim id est laborum.

Pada tahun \tP tercatat sebanyak 144.741 kunjungan di fasilitas layanan kesehatan di \nama_kabupaten. Sebanyak 123.456 kunjungan adalah ke fasilitas kesehatan milik pemerintah, sedangkan kunjungan ke fasilitas kesehatan milik swasta adalah sebanyak 78.901 kunjungan (\autoref{fig:Kunjungan-Faskes}).

\begin{figure}[H]
    \centering{}
    \includegraphics[width=0.8\textwidth]{bab_03/bab_03_01_kunjunganFaskes}
    \caption{Kunjungan Pasien Berdasarkan Jenis Faskes di \nama_kabupaten Tahun \tP}
    \label{fig:Kunjungan-Faskes}
\end{figure}

Pada tahun \tP tercatat sebanyak 12.345 kunjungan rawat inap dan 6.789 kunjungan rawat jalan di fasilitas layanan kesehatan di \nama_kabupaten. Sebanyak 12.000 kunjungan adalah di Fasilitas Kesehatan Tingkat Pertama, sedangkan kunjungan di Fasilitas Kesehatan Tingkat Lanjutan adalah sebanyak 3.456 kunjungan (\autoref{fig:Kunjungan-Rawat}).

\begin{figure}[H]
    \centering{}
    \includegraphics[width=0.8\textwidth]{bab_03/bab_03_02_rawat}
    \caption{Kunjungan Pasien Berdasarkan Jenis Perawatan di \nama_kabupaten Tahun \tP}
    \label{fig:Kunjungan-Rawat}
\end{figure}

\subsection{Kinerja pelayanan rumah sakit}
Kinerja pelayanan rumah sakit dapat dinilai berdasarkan beberapa indikator, antara lain:
\begin{itemize}
 \item \emph{Gross Death Rate}(GDR), yaitu angka kematian umum untuk tiap-tiap 1.000 pasien keluar;
 \item \emph{Net Death Rate} (NDR), yaitu angka kematian ≥ 48 jam setelah dirawat untuk tiap-tiap 1.000 pasien keluar;
 \item \emph{Bed Occupancy Rate} (BOR), yaitu persentase pemakaian tempat tidur pada satu-satuan waktu tertentu;
 \item \emph{Bed Turn Over} (BTO), yaitu frekuensi pemakaian tempat tidur pada satu periode, berapa kali tempat tidur dipakai dalam satu satuan waktu;
 \item \emph{Turn Over Interval} (TOI), yaitu rata-rata hari tempat tidur tidak ditempati dari saat terisi ke saat terisi berikutnya; dan
 \item \emph{Average Length of Stay} (ALOS), yaitu rata-rata lama rawat (dalam satuan hari) seorang pasien.
\end{itemize}

Kinerja pelayanan rumah sakit di \nama_kabupaten pada tahun \tP yang telah berada di dalam kondisi ideal adalah indikator NDR dan BTO (\autoref{tab:Kinerja-RS}).

\begin{table}[!ht]
\caption{Kinerja Pelayanan Rumah Sakit di \nama_kabupaten Tahun \tP}
\label{tab:Kinerja-RS}
\centering{}%
\ra{1.3}

\begin{tabular}{clcc}
\toprule
No & Indikator & Cakupan \tP & Kondisi Ideal\\
\midrule
1. & \emph{Gross Death Rate} & XX,XX per 1.000 & $\leq$ 45 per 1.000\\
\rowcolor{black!20}2. & \emph{Net Death Rate} & YY,YY per 1.000 & $\leq$ 25 per 1.000\\
3. & \emph{Bed Occupancy Rate} & ZZ,ZZ\% & 60\% - 80\%\\
\rowcolor{black!20}4. & \emph{Bed Turn Over} & XX,XX kali & 40 - 50 kali\\
5. & \emph{Turn Over Interval} & Y,YY hari & 1 - 3 hari\\
\rowcolor{black!20}6. & \emph{Average Length of Stay} & Z,ZZ hari & 6 - 9 hari\\
\bottomrule
\end{tabular}
\end{table}


\section[UKBM]{UPAYA KESEHATAN BERSUMBERDAYA MASYARAKAT (UKBM)}% for title and page
\sectionmark{UKBM} %for header
\subsection{Posyandu}
Lorem ipsum dolor sit amet, consectetur adipiscing elit, sed do eiusmod tempor incididunt ut labore et dolore magna aliqua. Ut enim ad minim veniam, quis nostrud exercitation ullamco laboris nisi ut aliquip ex ea commodo consequat. Duis aute irure dolor in reprehenderit in voluptate velit esse cillum dolore eu fugiat nulla pariatur. Excepteur sint occaecat cupidatat non proident, sunt in culpa qui officia deserunt mollit anim id est laborum.

Jumlah Posyandu di \nama_kabupaten tahun \tP adalah sebanyak XX posyandu aktif dari total YY unit posyandu (\autoref{tab:Posyandu-Posbindu-di-Kab}).

\begin{table}[!h]
\caption{Jumlah Posyandu dan Posbindu PTM di \nama_kabupaten Tahun \tP }
\label{tab:Posyandu-Posbindu-di-Kab}
\centering{}%
\ra{1.3}

\begin{tabular}{clrrr}
\toprule
No & Kecamatan & \multicolumn{1}{c}{Posyandu} & \multicolumn{1}{c}{Posyandu Aktif} & \multicolumn{1}{c}{Posbindu PTM}\\
\midrule
1. & Kabupaten A & X & Y & Z\\
\rowcolor{black!20}2. & Kabupaten B & X & Y & Z\\
3. & Kabupaten B & X & Y & Z\\
\midrule
\rowcolor{blue!20}\multicolumn{2}{c}{Jumlah} & X & Y & Z\\
\bottomrule
\end{tabular}
\end{table}

\subsection{Posbindu PTM}
Curabitur pretium tincidunt lacus. Nulla gravida orci a odio. Nullam varius, turpis et commodo pharetra, est eros bibendum elit, nec luctus magna felis sollicitudin mauris. Integer in mauris eu nibh euismod gravida. Duis ac tellus et risus vulputate vehicula. Donec lobortis risus a elit. Etiam tempor. Ut ullamcorper, ligula eu tempor congue, eros est euismod turpis, id tincidunt sapien risus a quam. Maecenas fermentum consequat mi. 

Jumlah Posbindu PTM di \nama_kabupaten tahun \tP adalah sebanyak 62 Posbindu PTM (\autoref{tab:Posyandu-Posbindu-di-Kab}).

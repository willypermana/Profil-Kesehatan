\usepackage{fontspec}
%\addfontfeature{LetterSpace=1.0}
\defaultfontfeatures{Ligatures=TeX}
\setmainfont{Roboto}[Mapping=tex-text]
\usepackage[paperwidth=215mm,paperheight=330mm,tmargin=1in,bmargin=1in,lmargin=1.5in,rmargin=1in,driver=xetex,verbose]{geometry}
%\geometry{verbose,tmargin=1in,bmargin=1in,lmargin=1.5in,rmargin=1in}
\usepackage{fancyhdr}
\pagestyle{fancy}
\setcounter{secnumdepth}{4}
\setcounter{tocdepth}{4}
\usepackage{array}
\usepackage{longtable}
\usepackage{refstyle}
\usepackage{float}
\usepackage{todonotes}
\reversemarginpar % place todo notes on left margin
\usepackage{graphicx}
\usepackage{microtype}
\usepackage[toc, page]{appendix}
\usepackage{pdfpages}
% necessary in two side book to remove header from empty page
\usepackage{emptypage}
%need to be installed
%refstyle todonotes appendix tocbibind multirow makecell footmisc metalogo titlesec
\makeatletter

%%%%%%%%%%%%%%%%%%%%%%%%%%%%%% LyX specific LaTeX commands.

\AtBeginDocument{\providecommand\figref[1]{\ref{fig:#1}}}
%% Because html converters don't know tabularnewline
\providecommand{\tabularnewline}{\\}
\RS@ifundefined{subsecref}
  {\newref{subsec}{name = \RSsectxt}}
  {}
\RS@ifundefined{thmref}
  {\def\RSthmtxt{theorem~}\newref{thm}{name = \RSthmtxt}}
  {}
\RS@ifundefined{lemref}
  {\def\RSlemtxt{lemma~}\newref{lem}{name = \RSlemtxt}}
  {}


%%%%%%%%%%%%%%%%%%%%%%%%%%%%%% Textclass specific LaTeX commands.
\newlength{\lyxlabelwidth}      % auxiliary length
\newcommand{\lyxaddress}[1]{
\par {\raggedright #1
\vspace{1.4em}
\noindent\par}
}

%%%%%%%%%%%%%%%%%%%%%%%%%%%%%% User specified LaTeX commands.
%\setromanfont{Lora}[Mapping=tex-text]
%% in XeTeX the fonstspec must load OpenType fonts present within Tex distribution by their file name
%% therefore the bold and italic font must also be declared separately
%\setromanfont{texgyrepagella-regular.otf}[BoldFont={texgyrepagella-bold.otf},ItalicFont={texgyrepagella-italic.otf}, Mapping=tex-text]
%% disregrad the above. XeTex will use any fonts that's installed system-wide.
\setromanfont{Roboto}[Mapping=tex-text]
\setsansfont{Roboto}[Mapping=tex-text]
\setmonofont{Envy Code R}
\tolerance=270
\emergencystretch=1.5em
% additional fonts settings
\usepackage{textcomp}
\usepackage{amssymb}
%\usepackage{fontawesome}
\usepackage{xunicode}

%% remove date in title
\date{\vspace{-5ex}}

\@ifundefined{showcaptionsetup}{}{%
 \PassOptionsToPackage{caption=false}{subfig}}
\usepackage{tocbibind}
\usepackage{subcaption}
\usepackage{booktabs}
\usepackage{colortbl}
\usepackage{xcolor}
\usepackage{multirow}
\usepackage{makecell}
\usepackage{pdflscape}
\usepackage[bottom]{footmisc} %fix footnote position
\usepackage{rotating} %for sideways
\usepackage{metalogo} %for xelatex logo
\usepackage[font={small,it}]{caption}
\makeatother
\usepackage{fancybox}
%\usepackage[raggedright,bf,sf,toctitles]{titlesec}
\usepackage[]{titlesec}
\definecolor{gray75}{gray}{0.75}
\newcommand{\hsp}{\hspace{20pt}}
\titleformat{\chapter}[hang]{\Huge\bfseries}{\thechapter\hsp\textcolor{gray75}{|}\hsp}{0pt}{\Huge\bfseries\raggedright}

%% redefine structure names
\def\subsectionautorefname{Subseksi}
\def\figureautorefname{Gambar}%
\def\tableautorefname{Tabel}%
\renewcommand\appendixpagename{Lampiran}

%% additional tabular settings
\newcommand{\ra}[1]{\renewcommand{\arraystretch}{#1}}
%\newcommand{\ra}[1]{\renewcommand{\arraystretch}{#1}}
\renewcommand{\tabcolsep}{7pt}

%% column definition for sum and percentage columns
\newcolumntype{x}{>{\centering\let\newline\\\arraybackslash\hspace{0pt}}p{}}
\newcolumntype{X}[1]{>{\centering\let\newline\\\arraybackslash\hspace{0pt}}p{#1}}
\newcolumntype{Y}[1]{>{\raggedright\let\newline\\\arraybackslash\hspace{0pt}}p{#1}}
%\newcolumntype{C}[1]{>{\centering\let\newline\\\arraybackslash\hspace{0pt}}m{#1}}
\newcolumntype{Z}[1]{>{\raggedleft\let\newline\\\arraybackslash\hspace{0pt}}p{#1}}

%% just trying something fancy
\usepackage{epigraph} 

%% Indonesian localization
%\usepackage[bahasai]{babel}
\usepackage{polyglossia} % better than babel package
%\setdefaultlanguage{bahasai} % default to Indonesian variant
\setmainlanguage[variant=indonesian]{malay} % {bahasai} sometimes doesn't work

\usepackage{wrapfig}

\usepackage{tikz}
\usetikzlibrary{shadows,calc}

% code adapted from https://tex.stackexchange.com/a/11483/3954
% some parameters for customization
\def\shadowshift{3pt,-3pt}
\def\shadowradius{6pt}

\colorlet{innercolor}{black!60}
\colorlet{outercolor}{gray!05}

% this draws a shadow under a rectangle node
\newcommand\drawshadow[1]{
    \begin{pgfonlayer}{shadow}
        \shade[outercolor,inner color=innercolor,outer color=outercolor] ($(#1.south west)+(\shadowshift)+(\shadowradius/2,\shadowradius/2)$) circle (\shadowradius);
        \shade[outercolor,inner color=innercolor,outer color=outercolor] ($(#1.north west)+(\shadowshift)+(\shadowradius/2,-\shadowradius/2)$) circle (\shadowradius);
        \shade[outercolor,inner color=innercolor,outer color=outercolor] ($(#1.south east)+(\shadowshift)+(-\shadowradius/2,\shadowradius/2)$) circle (\shadowradius);
        \shade[outercolor,inner color=innercolor,outer color=outercolor] ($(#1.north east)+(\shadowshift)+(-\shadowradius/2,-\shadowradius/2)$) circle (\shadowradius);
        \shade[top color=innercolor,bottom color=outercolor] ($(#1.south west)+(\shadowshift)+(\shadowradius/2,-\shadowradius/2)$) rectangle ($(#1.south east)+(\shadowshift)+(-\shadowradius/2,\shadowradius/2)$);
        \shade[left color=innercolor,right color=outercolor] ($(#1.south east)+(\shadowshift)+(-\shadowradius/2,\shadowradius/2)$) rectangle ($(#1.north east)+(\shadowshift)+(\shadowradius/2,-\shadowradius/2)$);
        \shade[bottom color=innercolor,top color=outercolor] ($(#1.north west)+(\shadowshift)+(\shadowradius/2,-\shadowradius/2)$) rectangle ($(#1.north east)+(\shadowshift)+(-\shadowradius/2,\shadowradius/2)$);
        \shade[outercolor,right color=innercolor,left color=outercolor] ($(#1.south west)+(\shadowshift)+(-\shadowradius/2,\shadowradius/2)$) rectangle ($(#1.north west)+(\shadowshift)+(\shadowradius/2,-\shadowradius/2)$);
        \filldraw ($(#1.south west)+(\shadowshift)+(\shadowradius/2,\shadowradius/2)$) rectangle ($(#1.north east)+(\shadowshift)-(\shadowradius/2,\shadowradius/2)$);
    \end{pgfonlayer}
}

% create a shadow layer, so that we don't need to worry about overdrawing other things
\pgfdeclarelayer{shadow}
\pgfsetlayers{shadow,main}

\newsavebox\mybox
\newlength\mylen

\newcommand\shadowimage[2][]{%
\setbox0=\hbox{\includegraphics[#1]{#2}}
\setlength\mylen{\wd0}
\ifnum\mylen<\ht0
\setlength\mylen{\ht0}
\fi
\divide \mylen by 120
\def\shadowshift{\mylen,-\mylen}
\def\shadowradius{\the\dimexpr\mylen+\mylen+\mylen\relax}
\begin{tikzpicture}
\node[anchor=south west,inner sep=0] (image) at (0,0) {\includegraphics[#1]{#2}};
\drawshadow{image}
\end{tikzpicture}} 

% dealing with two-side whitespaces
\raggedbottom
\usepackage[bottom]{footmisc}

% generate bibliography
\usepackage[backend=biber,sorting=nyt,style=numeric]{biblatex}
\addbibresource{pustaka.bib}

% load hyperref last
\usepackage[unicode=true,
bookmarks=true, bookmarksnumbered=true, bookmarksopen=false,
breaklinks=true, pdfstartview={0 0 1}, pdfborder={0 0 0}, pdfborderstyle={}, linktocpage=true]{hyperref}
\hypersetup{pdftitle={PDF title},
	pdfauthor={Dinas Kesehatan}}
% this is kinda necessary. I forgot why though
\phantomsection
% add \protect\numberline{} to get nice indentation in the List of Tables
\addcontentsline{lot}{section}{\protect\numberline{}Tabel 1 - Luas wilayah, jumlah desa/ kelurahan, jumlah penduduk, jumlah rumah tangga dan kepadatan penduduk}
\label{tabel-01}
\ra{1.3}

% Table generated by Excel2LaTeX from sheet '1'
{\centering
\begin{tabular}{clrrrrrrrr}
    \multicolumn{10}{l}{Tabel 1}\\
    \multicolumn{10}{c}{LUAS WILAYAH, JUMLAH DESA/ KELURAHAN, JUMLAH PENDUDUK, JUMLAH RUMAH TANGGA,}\\
    \multicolumn{10}{c}{DAN KEPADATAN PENDUDUK MENURUT KECAMATAN.}\\
    \multicolumn{10}{c}{\namaKabupatenKapital}\\
    \multicolumn{10}{c}{TAHUN \tP}\\
    \toprule
    \multicolumn{1}{c}{\multirow{2}[0]{*}{NO}} & \multicolumn{1}{c}{\multirow{2}[0]{*}{KECAMATAN}} & \multicolumn{1}{c}{\multirow{2}[0]{*}{\parbox{6em}{\centering LUAS WILAYAH (Km\textsuperscript{2})}}} & \multicolumn{3}{X{16em}}{JUMLAH} & \multicolumn{1}{c}{\multirow{2}[0]{*}{\parbox{6em}{\centering JUMLAH PENDUDUK}}} & \multicolumn{1}{c}{\multirow{2}[0]{*}{\parbox{6em}{\centering JUMLAH RUMAH TANGGA }}} & \multicolumn{1}{c}{\multirow{2}[0]{*}{\parbox{6em}{\centering RATA-RATA JIWA/ RUMAH TANGGA }}} & \multicolumn{1}{c}{\multirow{2}[0]{*}{\parbox{6em}{\centering KEPADATAN PENDUDUK PER Km\textsuperscript{2}}}} \\
    \cmidrule{4-6}
    & & & \multicolumn{1}{X{5em}}{DESA } & \multicolumn{1}{X{5em}}{KELURAHAN} & \multicolumn{1}{X{6em}}{DESA + KELURAHAN} & & & & \\
    \midrule
    \multicolumn{1}{c}{\emph{1}} & \multicolumn{1}{c}{\emph{2}} & \multicolumn{1}{c}{\emph{3}} & \multicolumn{1}{c}{\emph{4}} & \multicolumn{1}{c}{\emph{5}} & \multicolumn{1}{c}{\emph{6}} & \multicolumn{1}{c}{\emph{7}} & \multicolumn{1}{c}{\emph{8}} & \multicolumn{1}{c}{\emph{9}} & \multicolumn{1}{c}{\emph{10}}\\
    \midrule
	1 & Kecamatan A & xxx & y & z & x & yy.yyy & zz.zzz & x,xx & yyy,yy\\
	2 & Kecamatan B & xxx & y & z & x & yy.yyy & zz.zzz & x,xx & yyy,yy\\
	3 & Kecamatan C & xxx & y & z & x & yy.yyy & zz.zzz & x,xx & yyy,yy\\
	4 & Kecamatan ... & xxx & y & z & x & yy.yyy & zz.zzz & x,xx & yyy,yy\\
    \midrule
    \multicolumn{2}{l}{JUMLAH KAB.}& xxx & y & z & x & yy.yyy & zz.zzz & x,xx & yyy,yy\\
    \bottomrule
\end{tabular}%

}

\vfill
% Either Capil or BPS
Sumber: Dinas Kependudukan dan Pencatatan Sipil \namaKabupaten \par 

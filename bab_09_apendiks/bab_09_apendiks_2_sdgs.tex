% Table generated by Excel2LaTeX from sheet 'Sheet1'
\chapter{Sustainable Development Goals (SDGs)}
\begin{center}
    \renewcommand*{\arraystretch}{1.3}
    \begin{longtable}{rY{12em}rrrY{6em}}
        % prevent longtable show up twice in TOC. set caption for firsthead, then set empty starred
        % caption for next heads. in longtable \toprule should be prepended by \\
        \caption{Capaian \emph{Sustainable Development Goals} (SDGs) Bidang Kesehatan \namaKabupaten Tahun \tP}
        \\ \toprule
        No & Target/ Indikator SDGs & Pembilang & Penyebut & Capaian \tP & Satuan\\
        \midrule
        \endfirsthead
        \caption*{}
        \\ \toprule
        No & Target/ Indikator SDGs & Pembilang & Penyebut & Capaian \tP & Satuan\\
        \midrule
        \endhead
        % make bottom border on when table continue to next page
        \hline
        \endfoot
        \endlastfoot
                   1 & Proporsi peserta jaminan kesehatan melalui SSJN Bidang Kesehatan                                                               & 126.101 & 128.621 &    98,04 & \%                  \\
                   \rowcolor{black!5}2 & Persentase perempuan pernah kawin umur 15-49 tahun yang proses melahirkan terakhirnya di fasilitas kesehatan                   &   1.784 &   2.316 &    77,03 & \%                  \\
                   3 & Persentase anak umur 12-23 bulan yang menerima imunisasi dasar lengkap                                                         &   2.076 &   2.063 &   100,63 & \%                  \\
                   \rowcolor{black!5}4 & Prevalensi penggunaan metode kontrasepsi (CPR) semua cara pada Pasangan Usia Subur (PUS) usia 15-49 tahun yang berstatus kawin &  17.221 &  21.866 &    78,76 & \%                  \\
                   5 & Prevalensi kekurangan gizi (underweight) pada anak balita                                                                      &     488 &   8.129 &     6,00 & \%                  \\
                   \rowcolor{black!5}6 & Proporsi penduduk dengan asupan kalori minimum 1400 kkal/kapita/hari                                                           &         &         &      N/A & \%                  \\
                   7 & Prevalensi stunting (pendek dan sangat pendek) pada anak di bawah lima tahun/ balita                                           &     365 &   8.116 &     4,50 & \%                  \\
                   \rowcolor{black!5} 8 & Prevalensi stunting (pendek dan sangat pendek) pada anak di bawah dua tahun/ baduta                                            &         &         & \#DIV/0! & \%                  \\
                   9 & Prevalensi malnutrisi (berat badan/ tinggi badan) pada anak kurang dari lima tahun berdasarkan tipe                            &     190 &   8.077 &     2,35 & \%                  \\
                   \rowcolor{black!5}10 & Prevalensi anemia pada ibu hamil                                                                                               &         &         & 4,49 & \%                  \\
                   11 & Persentase bayi usia kurang dari 6 bulan yang mendapatkan ASI eksklusif                                                        &     973 &   1.589 &    61,23 & \%                  \\
                   \rowcolor{black!5}12 & Persentase perempuan pernah kawin umur 15-49 tahun yang proses melahirkan terakhirnya ditolong oleh tenaga kesehatan terlatih  &   1.784 &   2.316 &    77,03 & \%                  \\
                   13 & Angka Kematian Balita (AKBa) per 1.000 kelahiran hidup                                                                         &      26 &   1.788 &    14,54 & /1.000KH            \\
                   \rowcolor{black!5}14 & Angka Kematian Neonatal (AKN) per 1.000 kelahiran hidup                                                                        &      15 &   1.788 &     8,39 & /1.000KH            \\
                   15 & Angka Kematian Bayi (AKB) per 1.000 kelahiran hidup                                                                            &      22 &   1.788 &     1,23 & /1.000KH            \\
                   \rowcolor{black!5}16 & Persentase kabupaten/ kota yang mencapai 80\% imunisasi dasar lengkap pada bayi                                                &       1 &       1 &   100,00 & \%                  \\
                   17 & Prevalensi HIV/AIDS pada populasi dewasa                                                                                       &      12 &  97.329 &    0,012 & \%                  \\
                   \rowcolor{black!5}18 & Insiden Tuberkulosis (TB) per 100.000 penduduk                                                                                 &     155 & 128.621 &   120,51 & /100.000            \\
                   19 & Kejadian Malaria per 1.000 orang                                                                                               &       0 &       0 &     0,00 & /1.000              \\
                   \rowcolor{black!5}20 & Jumlah kabupaten/ kota yang mencapai eliminasi malaria                                                                         &       1 &       1 &   100,00 & Kab.                \\
                   21 & Persentase kabupaten/ kota yang memerlukan deteksi dini untuk infeksi Hepatitis B                                              &       1 &       1 &   100,00 & \%                  \\
                   \rowcolor{black!5}22 & Jumlah orang yang memerlukan intervensi terhadap penyakit tropis yang terabaikan (Filariasis dan Kusta)                        &         &         &       19 & orang               \\
                   23 & Jumlah orang yang memerlukan intervensi terhadap penyakit tropis yang terabaikan (Kusta)                                       &         &         &        5 & orang               \\
                   \rowcolor{black!5}24 & Jumlah provinsi dengan eliminasi kusta                                                                                         &         &         &        0 & Kab.                \\
                   25 & Jumlah kabupaten/ kota dengan eliminiasi filariasis (berhasil lolos dalam survei penilaian transmisi tahap II)                 &         &         &        0 & Kab.                \\
                   \rowcolor{black!5}26 & Prevalensi tekanan darah tinggi                                                                                                &         &         &    29,51 & \%                  \\
                   27 & Prevalensi obesitas pada penduduk umur >=18 tahun                                                                              &         &         &      N/A & \%                  \\
                   \rowcolor{black!5}28 & Jumlah kabupaten/ kota yang memiliki puskesmas yang menyelenggarakan upaya kesehatan jiwa                                      &         &         &        1 & Kab.                \\
                   29 & Angka penggunaan metode kontrasepsi jangka panjang (MKJP) cara modern                                                          &  17.221 &  21.866 &    78,76 & \%                  \\
                   \rowcolor{black!5}30 & Angka kelahiran pada perempuan umur 15-19 tahun (ASFR)                                                                         &         &         &    30,56 & per 1.000 perempuan \\
                   31 & Total Fertility Rate (TFR)                                                                                                     &         &         &     1,74 & per perempuan       \\
                   \rowcolor{black!5}32 & Unmet need pelayanan kesehatan                                                                                                 &         &         &      N/A & \%                  \\
                   33 & Jumlah penduduk yang dicakup asuransi kesehatan atau sistem kesehatan masyarakat per 1.000 penduduk                            & 126.101 & 128.621 &   980,41 & /1.000 pddk         \\
                   \rowcolor{black!5}34 & Cakupan Jaminan Kesehatan Nasional (JKN)                                                                                       & 126.101 & 128.621 &    98,04 & \%                  \\
                   35 & Persentase ketersediaan obat dan vaksin di Puskesmas                                                                           &         &         &   100,00 & \%                  \\
                   \rowcolor{black!5}36 & Unmet need KB (kebutuhan Keluarga Berencana/ KB yang tidak terpenuhi)                                                          &         &         & 12,59 & \%                  \\
                   37 & Pengetahuan dan pemahaman Pasangan Usia Subur (PUS) tentang metode kontrasepsi modern                                          &         &         & N/A & \%                  \\
                   \rowcolor{black!5}38 & Jumlah desa/ kelurahan yang melaksanakan Sanitasi Total Berbasis Masyarakat (STBM)                                             &         &         &        0 & Desa                \\
                   39 & Jumlah desa/ kelurahan yang Open Defecation Free (ODF)/ Stop Buang Air Besar Sembarangan (SBS)                                 &         &         &       39 & Desa \\
\bottomrule
\end{longtable}
\par\end{center}


